\documentclass[11pt]{article}
%\usepackage[firstpage]{draftwatermark}
\usepackage{times}
\usepackage{pdfpages}
\usepackage{fullpage}
\usepackage{url}
\usepackage{hyperref}
\usepackage{fancyhdr}
\usepackage{graphicx}
\usepackage{tabularx}
\usepackage{enumitem}
\usepackage{indentfirst}
\usepackage{subcaption}
\usepackage{amsmath, amsfonts, amsthm, fouriernc}
\usepackage{units}
\usepackage{IEEEtrantools}
\usepackage{parskip}
%\usepackage{multicol}
\usepackage{paracol}

\usepackage{color,soul}
\DeclareRobustCommand{\hlr}[1]{{\sethlcolor{red}\hl{#1}}}
\DeclareRobustCommand{\hlg}[1]{{\sethlcolor{green}\hl{#1}}}
\DeclareRobustCommand{\hlb}[1]{{\sethlcolor{blue}\hl{#1}}}
\DeclareRobustCommand{\hly}[1]{{\sethlcolor{yellow}\hl{#1}}}

\setcounter{secnumdepth}{4}
\graphicspath{{images/}}
\pagestyle{fancy}

\newenvironment{xitemize}{\begin{itemize}\addtolength{\itemsep}{-0.75em}}{\end{itemize}}
\newenvironment{tasklist}{\begin{enumerate}[label=\textbf{\thesubsubsection-\arabic*},ref=\thesubsubsection-\arabic*,leftmargin=*]}{\end{enumerate}}
\newcommand\todo[1]{{\bf TODO: #1}}
\setcounter{tocdepth}{2}
\setcounter{secnumdepth}{4}

\addtolength{\headheight}{2em}
\addtolength{\headsep}{1.5em}
\rhead{\large{ME 2801}}
\lhead{}
% This command sets the font size for all \section headings.
% 12 is the font size in points
% 15 is the baseline skip (vertical space between lines) in points
\usepackage{sectsty}
\sectionfont{\fontsize{12}{8}\selectfont}

\begin{document}

\begin{center}
  \Large{\bf{HW 5: Steady-State Error}}
\end{center}

The textbook exercises, designated by with the word “Nise”, are from “Problems” section at the end of the chapter in “Control System Engineering” by Norman Nise, 7th edition.

This assignment uses exercises from the textbook.  The additional instructions are to clarify and/or expand each exercise.

\section*{Nise 7.5}
Topic: Given the open-loop transfer function, predict the steady-state error for step (position), ramp (velocity) and parabolic (acceleration) test inputs.  By inspection of $G(s)$ you should be able to anticipate the type of steady-state error (zero, constant or infinite) for each of the test inputs.

\section*{Nise 7.10}

Use the \emph{Key Equations} (inside back cover of Nise textbook) to find the answers by hand. Optionally, you can check the answers using MATLAB.

\section*{Nise 7.17}
This exercise is meant to encourage revisiting the method used to derive the steady-state error equations for a step, ramp and parabola.  

\section{USV Case Study}

From your previous assignment we developed two proportional-only unity-feedback controllers: one for surge and one for yaw and evaluate the closed-loop response for step inputs in each case (surge and yaw).  You may have noticed that the P-only control hand different steady-state error properties for each degree-of-freedom.   In this exercise we'll use the steady-state error methods to better explain what we saw empirically.

\subsection*{Recall from last week}

Surge plant/process (open-loop) model:
\[
G_u(s)=\frac{ U(s) }{ T(s) } = \frac{ 1 }{ Ms+D } 
\]
for velocity as a function of thrust. Proportional feedback control gain: $K_u$.

Yaw plant/process (open-loop) model:
\[
G_r(s) = \frac{ R(s) }{ \tau (s) } = \frac{ 1 }{ s \, (Is+d) } 
\]
for yaw (heading) as a function of torque. Proportional feedback control gain $K_r$.

\subsection*{Exercises}
Consider each degree-of-freedom (surge and yaw) independently, i.e., work through the following steps twice: once for the surge velocity and once for the yaw position.

\begin{itemize}
  \item What is the system Type for each case?
  \item Based on the system Type, what type of input to the system would generate a non-zero constant steady-state error,  step (position), ramp (velocity) or parabola (acceleration) input?
  \item Consider the input type (step, ramp or parabola) that generates the non-zero constant steady-state error for the system.  Find an expression for this constant steady-state error  expressed in terms of pertinent variables, e.g., $K_u, K_r, M, D, I \text{and} d$.
  \begin{itemize}
    \item How do the physical model parameters affect (increase or decrease) steady-state error?
    \item How do the control gain parameters affect steady-state error?
  \end{itemize}
\end{itemize}

\end{document}


